% Lecture for ph2a Caltech 2017: Vibrations and Waves
\documentclass[pdf,hideothersubsections]{beamer}
\usepackage{beamerthemeshadow}
\mode<presentation>
  {
    \usefonttheme{structuresmallcapsserif}
    \usetheme{PaloAlto}
    \usecolortheme{seagull}
    %\useinnertheme{circles}
%    \useoutertheme{tree}
  }

\usepackage{svg}
\usepackage{xmpmulti}

\usepackage{hyperref}
\hypersetup{
    pdffitwindow=true,     % window fit to page when opened
    colorlinks=true,       % false: boxed links; true: colored links
    linkcolor=orange,          % color of internal links (change box color with linkbordercolor)
    citecolor=green,        % color of links to bibliography
    filecolor=magenta,      % color of file links
    urlcolor=blue           % color of external links
}

% Fonts/encoding
\renewcommand{\UrlFont}{\tiny}
\usepackage[utf8]{inputenc}
\usepackage[T1]{fontenc}
%\usepackage[sc,medium,raggedright]{titlesec}
\usepackage{newtxmath}
%\usepackage{libertine}
\usepackage[osf]{ebgaramond}

\graphicspath{{Figures/}}

\begin{document}
\title{Free Vibrations}  
\author{Caltech: ph2a}
\date{3 - Oct - 2017}
%\logo{\includegraphics[height=0.5cm]{../caltech_logo.png}}

\frame{\titlepage} 

\frame{\frametitle{Table of contents}\tableofcontents} 

\section{Previous Summary}
\begin{frame}
\frametitle{Last Week}
\begin{itemize}
\item Many physical systems involving small vibrations can be represented as a Simple Harmonic Oscillator.

\item Complex exponentials can be used to solve oscillator problems.

\item Guess solutions; set constants by setting (Dirichlet, Neumann, Cauchy) boundary conditions.

\item Imaginary part of the complex frequency corresponds to damping.
\end{itemize}
\end{frame}


\section{Damped Oscillations}
\begin{frame}
\frametitle{Complex approach to damped SHO}
Use the same approach as before:
\begin{itemize}
\item $L \ddot{Q} + R \dot{Q} + Q/C = 0$
\pause
\item $\tilde{Q}(t) = A e^{i (\omega t + \phi_0)}$
\pause
\item $-L \omega^2 \tilde{Q} + i \omega R \tilde{Q} + \tilde{Q}/C = 0$
\end{itemize}
\pause
Dividing by $L \tilde{Q}$, we get a quadratic equation for $\omega$: \\
\pause
\begin{center}
$\omega = \sqrt{\omega_0^2 - \Big(\frac{R}{2 L}\Big)^2} + i \frac{R}{2 L}$ \\
\end{center}
\pause
Let's call the real part $\omega$ and plug back into the guess for x(t):
\pause
\begin{center}
$\tilde{x} = A e^{i (\omega t + \phi_0)} e^{-(R/2 L) t}$\\
\end{center}
\pause
We can define the time constant: $\tau \equiv 2 L/R$, so that: \\
\pause
Taking the real part, we find the solution as before (with the damping): \\
\pause
\begin{center}
$Re\left\{\tilde{Q}\right\} = Q = A \cos(\omega t + \phi_o) e^{-t/\tau}$
\end{center}

\end{frame}

\subsection{Classes of Damping}
\begin{frame}
\frametitle{Under, Over, \& Critical Damping}

\begin{center}
$\omega = \sqrt{\omega_0^2 - \Big(\frac{R}{2 L}\Big)^2} + i \frac{R}{2 L}$ \\
\end{center}
\pause
\begin{block}{Definitions}
Under: $\omega_0^2 > \Big(\frac{R}{2 L}\Big)^2$ \\
Critical: $\omega_0^2 = \Big(\frac{R}{2 L}\Big)^2$ \\
Over: $\omega_0^2 < \Big(\frac{R}{2 L}\Big)^2$
\end{block}


\end{frame}


\subsection{Quality Factor}
\begin{frame}
\frametitle{The Quality Factor: I}

\begin{block}{Definition}
$Q \equiv 2 \pi \frac{\rm energy~stored}{\rm energy~dissipated~per~cycle}$
\end{block}

\end{frame}

\begin{frame}
\frametitle{Decaying Oscillation}

\centering
\includegraphics[width=0.5\textwidth]{damped_sine_wave.pdf}

\end{frame}

\begin{frame}
\frametitle{The Quality Factor: II}



\end{frame}


\subsection{Energy of Oscillators}
\begin{frame}
\frametitle{Kinetic and Potential Energy}


\end{frame}



\section{The Drag Force as Damping}
\begin{frame}
\frametitle{A Model for Drag Force}

\end{frame}




\section{Summary}
\begin{frame}
\frametitle{Summary}
\begin{enumerate}
\item For linear systems, the resultant of two vibrations is just their sum.
\pause
\item Summation of two oscillations with same period but different phase can be solved by vector addition of their complex phasors.
\pause
\item Sum of two oscillations with different frequencies leads to \emph{beats}.
\pause
\item Many non-linear systems can be analyzed by linearization.
\pause
\item Beating is apparent in physical systems even without non-linearity.
\end{enumerate}
\end{frame}


\end{document}

