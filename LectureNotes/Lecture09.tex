% Lecture for ph2a Caltech 2017: Vibrations and Waves
\documentclass[pdf, handout, hideothersubsections]{beamer}
\usepackage{beamerthemeshadow}
\mode<presentation>
  {
    \usefonttheme{structuresmallcapsserif}
    \usetheme{Darmstadt}
    \usecolortheme{whale}
    %\useinnertheme{circles}
%    \useoutertheme{tree}
  }

\usepackage{svg}
\usepackage{multimedia}
\usepackage{xmpmulti}
\usepackage{mathtools}
%\usepackage{enumitem}
\usepackage{hyperref}
\hypersetup{
    pdffitwindow=true,     % window fit to page when opened
    colorlinks=true,       % false: boxed links; true: colored links
    linkcolor=orange,      % color of internal links 
    citecolor=green,       % color of links to bibliography
    filecolor=magenta,     % color of file links
    urlcolor=blue,         % color of external links
    pdfstartview={Fit}
}

% Fonts/encoding
\renewcommand{\UrlFont}{\tiny}
\usepackage[utf8]{inputenc}
\usepackage[T1]{fontenc}
%\usepackage[sc,medium,raggedright]{titlesec}
\usepackage{newtxmath}
%\usepackage{libertine}
\usepackage[osf]{ebgaramond}

\graphicspath{{Figures/}}

\begin{document}
\title{Traveling Waves}  
\author{Caltech: ph2a}
\date{24 - Oct - 2017}


\frame{\titlepage} 

\frame{\frametitle{Table of Contents}\tableofcontents} 

\setbeamerfont{footnote}{size=\tiny}

\section{Summary}
\begin{frame}
\frametitle{Waves:}
\begin{enumerate}
  \pause
\item Restoring 'forces' lead to Partial Diff. Eq.
  \pause
\item Separable solutions yield ODE.
  \pause
\item This gives the wave equation: 
  $\ddot{y} = \frac{1}{v^2} \frac{\partial^2 y}{\partial t^2}$
  \pause
\item Second order diff. eq. results in oscillatory solutions.
\pause
\item Fourier series decomposition of solutions.
\pause
\item Including the time dependence can give us \emph{traveling}
  waves.
\pause
\item Examples: light, sound, radio, water, cell phone transmissions,
  gravitational waves, and gravity waves.

\end{enumerate}
\end{frame}


\begin{frame}
\frametitle{Overview}
\begin{enumerate}
\pause
\item Standing wave solutions are super-position of forward and
  backward traveling wave solutions.
\pause
\item Propagation of a pulse is governed by the \emph{wave equation}
\pause
\item Relationship between space and time arguments determines
  direction of propagation of the wave.

\end{enumerate}
\end{frame}


\section{Standing Waves}
\begin{frame}
\frametitle{The Wave Equation}
\begin{block}{The Wave Equation}
\centering
$ \frac{\partial^2 y}{\partial t^2} = \frac{T}{\mu} \frac{\partial^2 y}{\partial x^2}$
\end{block}
\pause
\begin{alignat}{2}
\frac{d^2 g}{dt^2} &= \omega^2 g  &\quad g(t) &= C_1 cos(\omega t) \\
\frac{d^2 f}{dx^2} &= -k^2 f &\quad f(x) &= A cos(k x) + B
sin(k x) 
\end{alignat}
\pause
where the velocity of the wave, $v \equiv \sqrt{\frac{T}{\mu}}$,
$\frac{\omega}{2 \pi}$ is the frequency, and $k \equiv \omega / v$ is
called the 'wave number'. For the wavelength of the wave, we use the
symbol $\lambda \equiv \frac{2 \pi}{k}$.
\end{frame}

\begin{frame}
\frametitle{Impose the Boundary Conditions}
\begin{centering}
\includegraphics[width=0.7\textwidth]{../../Simulations/modes_of_string.pdf}
\end{centering}
    \begin{enumerate}
    \item $f(x) = 0$ for $x \in \{0, L\} \implies$
      \pause
    \item $sin(k L) = 0$ for any and all t, and $A = 0$
      \pause
    \item so $k_n L = n \pi$ or $k_n = n \frac{\pi}{L}$
      \pause
    \item and $y(x, t) = A_n cos(\omega_n t) sin(k_n x)$
    \end{enumerate}
-- \href{http://www.baudline.com/}{Baudline} spectral analysis of
guitar, java string:
\end{frame}



\section{Traveling Waves}
\begin{frame}
\frametitle{Traveling (Progressive) Waves}
\pause
We can now use the trigonometric identity:
\begin{equation}
sin(\theta + \phi) + sin(\theta - \phi) = 2 sin(\theta) cos(\phi) 
\end{equation} \pause
to rewrite our general solution as:
\begin{align}
y(x, t) &= A_n cos(\omega_n t) sin(n \frac{\pi}{L} x) \\
        &= \frac{A_n}{2} \bigg[sin(k_n x - \omega_n t)  + sin(k_n x + \omega_n t)\bigg]
\label{eq:genstand}
\end{align}
\pause
with $k_n = n \pi/L$, $v = \frac{T}{\mu}$, and $\omega_n = n v \pi/L$,
\pause
where each term is a \emph{traveling} wave. \\


\end{frame}

\begin{frame}
\frametitle{Traveling Wave: Animation}
\begin{columns}
  \begin{column}{0.4\textwidth}
    \centering
    \includegraphics[height=5cm]{travwaves.pdf}
    % \movie[externalviewer, showcontrols=true, poster,
    % width=4cm]{}{../../Simulations/travelling_wave.mp4}
  \end{column}
\begin{column}{0.6\textwidth}
Questions:
\begin{enumerate}
\item The wave travels with a velocity $v$ in x-direction. What is the
  velocity of the individual points?
\item Do the points exhibit simple harmonic motion?
\item Do all of the points move with the same frequency?
\item Do all of the points move with the same amplitude?
\end{enumerate}
\end{column}
\end{columns}

\end{frame}

\begin{frame}
\frametitle{Traveling Waves: Details}
Let's examine Eq.~\ref{eq:genstand} in more detail:
\pause
\includegraphics[width=\textwidth]{../../Simulations/detail-trav.pdf}
\pause
similarly, $sin(k_n x + \omega_n t)$ represents a wave traveling to
the left.

\end{frame}

\begin{frame}
\frametitle{Properties of Traveling Waves}
\begin{enumerate}
\item Any value of $k = 2 \pi / \lambda$ and $\omega = k v$ are allowed. 
\pause
\item Infinite extent in space and time. No boundaries (in principle).
\pause
\item Principle of Superposition still applies: sum of traveling waves
  satisfies the wave equation.
\pause
\item Standing waves can be decomposed into a sum of traveling waves.
\pause
\item Traveling waves carry momentum and energy? How do we
  measure/calculate this?
\end{enumerate}
\end{frame}


\section{Summary}
\begin{frame}
\frametitle{Summary}
\begin{enumerate}
\pause
\item Standing wave solutions are super-position of forward and
  backward traveling wave solutions.

\item Propagation of a pulse is governed by the \emph{wave equation}

\item Relationship between space and time arguments determines
  direction of propagation of the wave.

\item Next: examine properties of air to examine sound waves and
  musical instruments (boundary conditions).

\end{enumerate}

\end{frame}

\end{document}

