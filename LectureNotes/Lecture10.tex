% Lecture for ph2a Caltech 2017: Vibrations and Waves
\documentclass[pdf, hideothersubsections]{beamer}
\usepackage{beamerthemeshadow}
\mode<presentation>
  {
    \usefonttheme{structuresmallcapsserif}
    \usetheme{CambridgeUS}
    \usecolortheme{seahorse}
    %\useinnertheme{circles}
%    \useoutertheme{tree}
  }

\usepackage{svg}
\usepackage{xmpmulti}
\usepackage{mathtools}
%\usepackage{enumitem}
\usepackage{hyperref}
\hypersetup{
    pdffitwindow=true,     % window fit to page when opened
    colorlinks=true,       % false: boxed links; true: colored links
    linkcolor=orange,      % color of internal links 
    citecolor=green,       % color of links to bibliography
    filecolor=magenta,     % color of file links
    urlcolor=blue,         % color of external links
    pdfstartview={Fit}
}

% Fonts/encoding
\renewcommand{\UrlFont}{\tiny}
\usepackage[utf8]{inputenc}
\usepackage[T1]{fontenc}
%\usepackage[sc,medium,raggedright]{titlesec}
\usepackage{newtxmath}
%\usepackage{libertine}
\usepackage[osf]{ebgaramond}

\graphicspath{{Figures/}}

\begin{document}
\title{Waving Strings}  
\author{Caltech: ph2a}
\date{19 - Oct - 2017}


\frame{\titlepage} 

\frame{\frametitle{Table of Contents}\tableofcontents} 

\setbeamerfont{footnote}{size=\tiny}

\section{Summary}
\begin{frame}
\frametitle{Last Time:}
\begin{enumerate}
  \pause
\item Matrix methods can be used to analyzed eigenmodes of linear systems.
  \pause
\item The case of $N$ springs and $N$ masses can be analyzed as a
  straightforward extension of the methods used for $N = 2, 3$.
  \pause
\item Strings are an extension of this method to $N \rightarrow \infty$.
  \pause
\item The Bode plot for systems with finite $N$, and then 
  infinite $N$, illustrates the phenomena of \emph{dispersion} and the
  \emph{cut-off} frequency.
\pause
\item The rest of this course is the study of waves. This week we make
  the gradual transition between vibrations of discrete elements to
  waves: vibrations in continuous systems.
\end{enumerate}
\end{frame}


\begin{frame}
\frametitle{Overview}

\end{frame}


\section{String with Fixed Ends}
\begin{frame}
\frametitle{String with Fixed Ends: Setup}

\end{frame}

\begin{frame}
\frametitle{Analogy with Masses/Springs}


\end{frame}


\begin{frame}
\frametitle{Details of the forces / tensions:}

\end{frame}



\section{Wave Equation: Solutions}
\begin{frame}
\frametitle{Wave Equation: Solutions}

\end{frame}


\begin{frame}
\frametitle{Wave Equation: Solutions}

\end{frame}

\begin{frame}
\frametitle{The ODE Solutions}

\end{frame}

\section{Normal Modes}
\begin{frame}
\frametitle{Normal Modes of a String}

\end{frame}

\begin{frame}
\frametitle{Normal Modes of a String}

\end{frame}

\section{Summary}
\begin{frame}
\frametitle{Summary}


\end{frame}

\end{document}

