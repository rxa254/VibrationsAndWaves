% Lecture for ph2a Caltech 2017: Vibrations and Waves
\documentclass[pdf, handout, hideothersubsections]{beamer}
\usepackage{beamerthemeshadow}
\mode<presentation>
  {
    \usefonttheme{structuresmallcapsserif}
    \usetheme{Darmstadt}
    \usecolortheme{whale}
    %\useinnertheme{circles}
%    \useoutertheme{tree}
  }

\usepackage{svg}
\usepackage{xmpmulti}
\usepackage{mathtools}
%\usepackage{enumitem}
\usepackage{hyperref}
\hypersetup{
    pdffitwindow=true,     % window fit to page when opened
    colorlinks=true,       % false: boxed links; true: colored links
    linkcolor=orange,      % color of internal links 
    citecolor=green,       % color of links to bibliography
    filecolor=magenta,     % color of file links
    urlcolor=blue,         % color of external links
    pdfstartview={Fit}
}

% Fonts/encoding
\renewcommand{\UrlFont}{\tiny}
\usepackage[utf8]{inputenc}
\usepackage[T1]{fontenc}
%\usepackage[sc,medium,raggedright]{titlesec}
\usepackage{newtxmath}
%\usepackage{libertine}
\usepackage[osf]{ebgaramond}

\graphicspath{{Figures/}}

\begin{document}
\title{Sonic Waves}  
\author{Caltech: ph2a}
\date{26 - Oct - 2017}


\frame{\titlepage} 

\frame{\frametitle{Table of Contents}\tableofcontents} 

\setbeamerfont{footnote}{size=\tiny}

\section{Summary}
\begin{frame}
\frametitle{Last Time:}
\begin{enumerate}
  \pause
\item The wave equation has solutions which are oscillations in space
  and time.
  \pause
\item Any arbitrary initial condition will result in a wave which can
  be decomposed into sines and cosines using Fourier analysis.
  \pause
\item Complex exponentials make solving the systems easier. We take
  the real part as usual for real solutions.
  \pause
\item The waves are indexed by their wavelength, $\lambda$, velocity,
  $v$, frequency, $f = \omega / (2 \pi)$, and wavenumber, $k = 2 \pi / \lambda$.
\pause
\item Traveling waves in 1D can be moving in either the +x or --x
  direction.
\pause
\item Standing waves can be generated by a superposition of traveling waves.
\end{enumerate}
\end{frame}


\section{The Springiness of Air}
\begin{frame}
\frametitle{What is Air made up of?}
molecules. What kind of molecules? How much of what kind of molecule?

The partial pressure of each of the major constituents of air
determines the properties of sound in air.


\end{frame}

\begin{frame}
\frametitle{What do we know about Sound?}
\begin{enumerate}
\item How does the speed of sound change as a function of frequency?
\pause Almost not at all. Words show up in the right order: $\omega(k)
=$~const.
\pause
\item How does the speed of sound change with temperature?
\pause
\item What about echoes? What absorbs and what reflects sound?
\pause
\item What attenuates sound? How far away can you hear a shout?
\pause
\item How many different ways can sound be generated?
\pause
\item How does sounds transfer energy? (think wineglass)

\end{enumerate}
\end{frame}


\begin{frame}
\frametitle{The Paradigm of Sound}
\centering
\includegraphics[width=0.7\columnwidth]{../../Simulations/PressureOsc.pdf}

\url{http://www.physicsclassroom.com/class/sound}
\end{frame}

\begin{frame}
\frametitle{Pressure Wave in Air}

\centering
\includegraphics[width=0.9\columnwidth]{sound-wave.png}

\end{frame}


\begin{frame}
\frametitle{The Mechanics of Sound}
\fontsize{8.5pt}{10.2}\selectfont
As a start, we analyze the motion of a piston\footnotemark in an air
chamber (i.e. a bicycle pump), but
this turns out to be a reasonable approximation for pressure waves in air.
\pause
\begin{columns}
\begin{column}{0.6\textwidth}
\begin{enumerate}
\item Assume we have a piston (with mass $m$ and cross-sectional area $A$) as shown on the right:
\pause
\item The ideal gas law is $p V = N k_B T$, where $k_B$ is Boltzmann's
  constant, $N$ is the number of particles, $T$ is the temperature of
  the gas, $V$ is the initial volume of the air in the piston, and $p$
  is the initial pressure of the gas.
\pause
\item Everything is at equilibrium, then we apply a force $F$. This
  results in change in pressure: $F = \Delta p \times A$.
\pause
\item If the change in the pressure is made \emph{slowly}, then the
  temperature of the gas stays the same through contact with the
  walls.
\pause
\item This is then an \emph{isothermal} change, and we see that $p V = $const.
\end{enumerate}
\end{column}

\begin{column}{0.4\textwidth}

\includegraphics[width=0.9\columnwidth]{piston.jpg}

\end{column}
\end{columns}
\footnote[1]{French: Ch.3 pp. 57--59}

\end{frame}


\begin{frame}
\frametitle{Isothermal Pressure change}
IF $p V = $~const, then, by the chain rule, $p \delta V + \delta p V = 0$.\\
\pause
\begin{align}
\delta p &= -p (\delta V)/V \\
F / A    &= -p (A y)/(A l) \\
       F &= - \frac{p A}{l} y
\end{align}
\pause
so now we have a restoring force with an effective spring constant 
$k = p A/l$. Once we have this spring constant (i.e. restoring force
depends linearly on the compression), we know we get SHO.
\end{frame}


\begin{frame}
\frametitle{Adiabatic Compression}
However, for most sounds, the vibration frequency is high so $p V \ne $~const. i.e., the period of the sound wave is much, much less than
the time it takes for the volume of air to be thermalized with the environment. \\
\pause
For these adiabatic compressions: $p V^{\gamma} = \mathrm{const}$. \\
\pause
Following the same analysis as for the isothermal case\footnotemark, we find
$k_{adiabat}=\gamma k_{isotherm}$.\\
\pause
$\gamma$ depends on the specific heat of the gas, which in turn
depends on the number of rotational and vibrational degrees of freedom
of the gas. \pause So $\gamma = 1.67$ for mono-atomic gases, 1.4 for diatomic
gases (e.g. H$_2$, O$_2$, N$_2$), and $\sim$10--20\% smaller for more
complicated molecules.

\footnote[1]{French Ch. 6 pp. 176--178}
\end{frame}


\begin{frame}
\frametitle{Towards a Wave Equation for Gas}
As before, with strings, we can write down a wave equation:
\begin{equation}
\frac{\partial^2 z}{\partial t^2} = \frac{1}{v^2}\frac{\partial^2 z}{\partial x^2}
\end{equation}
where $z$ is the displacement of the gas in the $x$
direction. i.e. $z(x, t)$ represents the displacement of the gas as a
function of its initial position $x$.


\centering
\includegraphics[width=0.7\columnwidth]{sound-wave.png}

French Ch. 7 pp. 209--212
\end{frame}


\begin{frame}
\frametitle{Towards a Wave Equation for Gas}
To describe the wave we will need to know the velocity.
\begin{equation}
\frac{\partial^2 z}{\partial t^2} = \frac{1}{v^2}\frac{\partial^2 z}{\partial x^2}
\end{equation}
Following the description for speed of waves in a solid rod (Ch. 6),
we find $v = \frac{\gamma p}{\rho}$, where $\rho$ is the density of
air ($\sim$1.2 kg/m$^3$ at sea level and room temperature).

\url{http://www.feynmanlectures.caltech.edu/I_47.html}
\end{frame}


\section{Standing Waves}
\begin{frame}
\frametitle{Standing Waves in a String}
\begin{block}{The Wave Equation}
\centering
$ \frac{\partial^2 y}{\partial t^2} = \frac{T}{\mu} \frac{\partial^2 y}{\partial x^2}$
\end{block}
\pause
\begin{alignat}{2}
\frac{d^2 g}{dt^2} &= \omega^2 g  &\quad g(t) &= C_1 cos(\omega t) \\
\frac{d^2 f}{dx^2} &= -k^2 f &\quad f(x) &= A cos(k x) + B
sin(k x) 
\end{alignat}
\pause
where the velocity of the wave, $v \equiv \sqrt{\frac{T}{\mu}}$,
$\frac{\omega}{2 \pi}$ is the frequency, and $k \equiv \omega / v$ is
called the 'wave number'. For the wavelength of the wave, we use the
symbol $\lambda \equiv \frac{2 \pi}{k}$.
\end{frame}

\begin{frame}
\frametitle{Impose the Boundary Conditions}
\begin{centering}
\includegraphics[width=0.7\textwidth]{../../Simulations/modes_of_string.pdf}
\end{centering}
    \begin{enumerate}
    \item $f(x) = 0$ for $x \in \{0, L\} \implies$
      \pause
    \item $sin(k L) = 0$ for any and all t, and $A = 0$
      \pause
    \item so $k_n L = n \pi$ or $k_n = n \frac{\pi}{L}$
      \pause
    \item and $y(x, t) = A_n cos(\omega_n t) sin(k_n x)$
    \end{enumerate}
\end{frame}



% \section{Wave Equation: Sonic Waves}
% \begin{frame}
% \frametitle{Standing Waves in a Pipe}


% \end{frame}


% \begin{frame}
% \frametitle{Wave Equation: Solutions}

% \end{frame}

% \begin{frame}
% \frametitle{The ODE Solutions}

% \end{frame}


% \section{}
% \begin{frame}
% \frametitle{Normal Modes of a String}

% \end{frame}

% \begin{frame}
% \frametitle{Normal Modes of a String}

% \end{frame}


% \section{Summary}
% \begin{frame}
% \frametitle{Summary}


%\end{frame}

\end{document}

