% Lecture for ph2a Caltech 2017: Vibrations and Waves
\documentclass[pdf, hideothersubsections]{beamer}
\usepackage{beamerthemeshadow}
\mode<presentation>
  {
    \usefonttheme{structuresmallcapsserif}
    \usetheme{Darmstadt}
    \usecolortheme{whale}
    %\useinnertheme{circles}
%    \useoutertheme{tree}
  }

\usepackage{svg}
\usepackage{xmpmulti}
\usepackage{mathtools}
%\usepackage{enumitem}
\usepackage{hyperref}
\hypersetup{
    pdffitwindow=true,     % window fit to page when opened
    colorlinks=true,       % false: boxed links; true: colored links
    linkcolor=orange,      % color of internal links 
    citecolor=green,       % color of links to bibliography
    filecolor=magenta,     % color of file links
    urlcolor=blue,         % color of external links
    pdfstartview={Fit}
}

% Fonts/encoding
\renewcommand{\UrlFont}{\ttfamily\tiny}
\usepackage[utf8]{inputenc}
\usepackage[T1]{fontenc}
%\usepackage[sc,medium,raggedright]{titlesec}
\usepackage{newtxmath}
%\usepackage{libertine}
\usepackage[osf]{ebgaramond}

\graphicspath{{Figures/}}

\begin{document}
\title{Interference}  
\author{Caltech: ph2a}
\date{21 - Nov - 2017}


\frame{\titlepage} 

\frame{\frametitle{Table of Contents}\tableofcontents} 

\setbeamerfont{footnote}{size=\tiny}

\section{Overview}
\begin{frame}
\frametitle{Overview:}
\begin{enumerate}
  \pause
\item Interference of waves leads to many phenomena: audio feedback
  for musicians, dead spots in cell reception, rogue waves in the
  ocean, satellite dishes, etc.
\pause
\item NASA Deep Space Network: communication with satellites
\pause
\item Optical Interferometry/Astronomy: Early Warnings for Extinction
  Level Asteroids
\pause
\item Time-tagged Cell phone accelerometer signals for early EQ warning

\end{enumerate}
\end{frame}

\begin{frame}
\frametitle{Satellite Dishes}
\includegraphics[width=0.99\textwidth]{Canberra_Deep_Space_Communication_Complex.jpg}

\end{frame}


\section{Interference of Two Waves}
\begin{frame}
\frametitle{Huygen's Principle}
\begin{columns}
  \begin{column}{0.3\textwidth}
    \includegraphics[width=0.99\textwidth]{Huygen.pdf}
  \end{column}

  \begin{column}{0.7\textwidth}
    \begin{itemize}
    \item A \emph{Wavefront} is a surface over which an optical wave has a
      constant phase.
      \pause
    \item In 3D, a point source emits spherical waves; i.e. it will emit
      spherical wavefronts.
      \pause
    \item For 2D, the wavefronts are circles.
      \pause
    \item Huygen's principle: the propagation of an arbitrary wave can be
      understood by imagining that each piece of a wavefront emits a
      secondary spherical wavelet. \pause The coherent superpostion of
      these wavelets determines the new wavefront.
    \end{itemize}
  \end{column}
\end{columns}
\end{frame}

\begin{frame}
\frametitle{Interference of Two Waves}
Let us consider two point sources of some kind of wave (need not be
electromagnetic):
\begin{columns}

  \begin{column}{0.4\textwidth}
    \includegraphics[width=0.99\textwidth]{TwoSource.jpg}
  \end{column}

  \begin{column}{0.6\textwidth}
\begin{align*}
\Psi &= \action<+->{A(r_1) e^{i (k r_1 - \omega t)} + A(r_2) e^{i (k r_2 - \omega t)}\\}
     \action<+->{&\simeq A \Big[e^{i (k r_1 - \omega t)} + e^{i (k r_2 - \omega t)}\Big] \\}
     \action<+->{&= A e^{i (k r - \omega t)} e^{i k (r_2 -
                   r_1)} \\}
\action<+->{|\Psi|^2 &= 4 |A|^2 cos^2[k (r_2 - r_1)] \\}
\action<+->{|\Psi|^2 &= 4 |A|^2 cos^2[2 \pi \frac{d}{\lambda} \sin{\theta}]}
\end{align*}
\pause
where $\delta \phi$ is the phase \emph{difference}, $d$ is the average
distance from the slits to the source, and $\theta$ is the angle to
the point on the screen.
  \end{column}

  \end{columns}
\end{frame}

\begin{frame}
\frametitle{2 Sources}

\centering
\includegraphics[width=0.85\textwidth]{../../Simulations/IMGintpattern_f5000_n2.png}

\end{frame}


\section{Interference of N Sources}
\begin{frame}
\frametitle{3 Sources}

\centering
\includegraphics[width=0.85\textwidth]{../../Simulations/IMGintpattern_f5000_n3.png}

\end{frame}

\begin{frame}
\frametitle{N Sources}

\centering
\includegraphics[width=0.99\textwidth]{../../Simulations/IMGintpattern_f5000_n9.png}


\end{frame}
%\section{Antennae}

\section{Interferometers}
\begin{frame}
\frametitle{Michelson Interferometer}
\begin{columns}
  \begin{column}{0.35\textwidth}
    \includegraphics[width=0.99\textwidth]{Michelson.jpg}
  \end{column}

  \begin{column}{0.65\textwidth}
    \begin{enumerate}
    \item Developed by Michelson in $\sim1880$ to look for the
      `luminiferous ether'
      \pause
    \item Illuminated by a single, `mono-chromatic' light source
      \pause
    \item Able to measure sub-nanometer displacements by $\sim1890$
      \pause
    \item Best Michelson interferometers today can measure
      $\sim10^{-16}$\,m.
      \pause
    \item Let's use our knowledge of superposition and interference to
      work out the sensitivity for two cases: \pause
      (1) the laser has truly one frequency, $\nu$ \pause
      and (2) the laser has a finite linewidth $\delta \nu$.
      
    \end{enumerate}
  \end{column}
\end{columns}
\end{frame}

\section{Optical Resonator}
\begin{frame}
\frametitle{The Fabry-Perot Interferometer}
Q: How can we determine the linewdith of a light (wave) source using
something other than a Michelson interferometer?\pause
A: We can use a SHO! Since we understand how the Bode Plot works for a
SHO, we can use it to characterize the laser. \pause

\begin{enumerate}
\item A Fabry-Perot resonator consists of two mirrors separated by a
  distance $L$, each with an \emph{amplitude} reflectivity, $r$.
\pause
\item As we did for the single layer anti-reflection coating, we can
  write down the relationships between all the fields.
\pause
\item Assuming that the mirrors are moving very slowly, with respect
  to the round trip travel time for the light, we can find the
  quasi-static solution to the fields.
\pause
\item Take a few minutes to solve for the fields...
\end{enumerate}
\pause
Now, let's plot it !!
\end{frame}


\begin{frame}
\frametitle{The Fabry-Perot Interferometer}

\centering
\includegraphics[width=0.99\textwidth]{../../Simulations/FabryPerot.pdf}

\end{frame}

\section{Summary}
\begin{frame}
\frametitle{Summary}
\begin{itemize}
  \item Huygen's principle can be used to propagate waves.
  \item It has not classical foundation, but it makes sense quantum-wise.
  \item 

  \item 
  \item 

  \item 

\end{itemize}
\footnotetext[1]{\url{https://www.khanacademy.org/test-prep/mcat/physical-processes/light-and-electromagnetic-radiation-questions/a/diffraction-and-constructive-and-destructive-interference}}
\end{frame}

\end{document}

