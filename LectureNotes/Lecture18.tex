% Lecture for ph2a Caltech 2017: Vibrations and Waves
\documentclass[pdf, handout, hideothersubsections]{beamer}
\usepackage{beamerthemeshadow}
\mode<presentation>
  {
    \usefonttheme{structuresmallcapsserif}
    \usetheme{Darmstadt}
    \usecolortheme{whale}
    %\useinnertheme{circles}
%    \useoutertheme{tree}
  }

\usepackage{svg}
\usepackage{xmpmulti}
\usepackage{mathtools}
%\usepackage{enumitem}
\usepackage{hyperref}
\hypersetup{
    pdffitwindow=true,     % window fit to page when opened
    colorlinks=true,       % false: boxed links; true: colored links
    linkcolor=orange,      % color of internal links 
    citecolor=green,       % color of links to bibliography
    filecolor=magenta,     % color of file links
    urlcolor=blue,         % color of external links
    pdfstartview={Fit}
}

% Fonts/encoding
\renewcommand{\UrlFont}{\ttfamily\tiny}
\usepackage[utf8]{inputenc}
\usepackage[T1]{fontenc}
%\usepackage[sc,medium,raggedright]{titlesec}
\usepackage{newtxmath}
%\usepackage{libertine}
\usepackage[osf]{ebgaramond}

\graphicspath{{Figures/}}

\begin{document}
\title{Interference \& Diffraction}  
\author{Caltech: ph2a}
\date{28 - Nov - 2017}


\frame{\titlepage} 

\frame{\frametitle{Table of Contents}\tableofcontents} 

\setbeamerfont{footnote}{size=\tiny}

\begin{frame}
\frametitle{Feynman}

\centering
``No one has ever been able to define the difference between
interference and diffraction satisfactorily.'' \\

-Feynman Lectures, 30-1

\end{frame}

\section{Interference of Two Sources}
\begin{frame}
\frametitle{Huygen's Principle: Review}
\begin{columns}
  \begin{column}{0.3\textwidth}
    \includegraphics[width=0.99\textwidth]{Huygen.pdf}
  \end{column}

  \begin{column}{0.7\textwidth}
    \begin{itemize}
    \item A \emph{Wavefront} is a surface over which an optical wave has a
      constant phase.

    \item In 3D, a point source emits spherical waves; i.e. it will emit
      spherical wavefronts.

    \item For 2D, the wavefronts are circles.

    \item Huygen's principle: the propagation of an arbitrary wave can be
      understood by imagining that each piece of a wavefront emits a
      secondary spherical wavelet.
      The coherent superpostion of
      these wavelets determines the new wavefront.
    \end{itemize}
  \end{column}
\end{columns}
\end{frame}

\begin{frame}
\frametitle{Interference of Two Waves}
Let us consider two point sources of some kind of wave (need not be
electromagnetic):
\begin{columns}

  \begin{column}{0.4\textwidth}
    \includegraphics[width=0.99\textwidth]{TwoSource.jpg}
  \end{column}

  \begin{column}{0.6\textwidth}
\begin{align*}
\Psi &= \action<+->{A(r_1) e^{i (k r_1 - \omega t)} + A(r_2) e^{i (k r_2 - \omega t)}\\}
     \action<+->{&\simeq A \Big[e^{i (k r_1 - \omega t)} + e^{i (k r_2 - \omega t)}\Big] \\}
     \action<+->{&= A e^{i (k r - \omega t)} \times 2 \cos{k (r_2 -
                   r_1)} \\}
\action<+->{|\Psi|^2 &= 4 |A|^2 cos^2[k (r_2 - r_1)] \\}
\action<+->{|\Psi|^2 &= 4 |A|^2 cos^2[2 \pi \frac{d}{\lambda} \sin{\theta}]}
\end{align*}

where $\delta \phi$ is the phase \emph{difference}, $d$ is the average
distance from the slits to the source, and $\theta$ is the angle to
the point on the screen.
  \end{column}

  \end{columns}
\end{frame}

\begin{frame}
\frametitle{2 Sources}

\centering
\includegraphics[width=0.85\textwidth]{../../Simulations/IMGintpattern_f5000_n2.png}

\end{frame}


\section{Interference of N Sources}
\begin{frame}
\frametitle{3 Sources}

\centering
\includegraphics[width=0.85\textwidth]{../../Simulations/IMGintpattern_f5000_n3.png}

\end{frame}


\begin{frame}
\frametitle{N Sources}

\centering
\includegraphics[width=0.85\textwidth]{../../Simulations/IMGintpattern_f5000_n9.png}

see animations...

\end{frame}

\begin{frame}
\frametitle{N Sources}

\begin{columns}
\begin{column}{0.5\textwidth}
\includegraphics[width=0.99\textwidth]{diffraction_grating_geometry.jpg}
\end{column}

\begin{column}{0.5\textwidth}
Constructive Interference: maxima at $m \lambda = d sin \theta$, where
$m = ..., -2, -1, 0, 1, 2,...$

Sketch of Derivation:\\
$\Psi(r, \theta, t) = A(r) e^{-i \omega t} \sum_{p=-N/2}^{N/2} e^{i k
  r_p}$ \\
$r_2 - r_1 = d \sin{\theta}$ \\
$r_3 - r_1 = 2 d \sin{\theta}$ \\
$r_4 - r_1 = 3 d \sin{\theta}$ \\
So, for the angle between maxima $\sin{\theta} = \lambda / d$. Or, for
the case where the screen is far away, $\theta \simeq \lambda / d$.
\end{column}

\end{columns}
\end{frame}

\begin{frame}
\frametitle{N Sources}

\begin{columns}
\begin{column}{0.5\textwidth}
\includegraphics[width=0.99\textwidth]{f30-01_tc_big.pdf}
\end{column}

\begin{column}{0.5\textwidth}
\begin{enumerate}
\item The wave from each slit has approximately the same amplitude, $A$.
\pause
\item The phase, $\phi$, from each slit is slightly retarded with respect to
  the adjacent one.
\pause
\item From trigonometry, we can see that $A = 2 r \sin{\phi/2}$
\pause
\item The angle $OQT = n \phi$, and so $A_R = 2 r \sin{n \phi / 2}$
\pause
\item So the amplitude $A_R = A \frac{\sin{n \phi /2}}{\sin{\phi/2}}$
    and
\pause
\item The intensity $I = I_0 \frac{\sin^2{n \phi /2}}{\sin^2{\phi/2}}$

\end{enumerate}
\end{column}

\end{columns}

\end{frame}

\begin{frame}
\frametitle{N Sources}

\centering
\includegraphics[width=0.85\textwidth]{../../Simulations/Nslits.pdf}


\end{frame}

%\section{Antennae}

\section{Interferometers}

\section{Optical Resonator / Etalon}
\begin{frame}
\frametitle{The Fabry-Perot Interferometer}
... review of last week's calculation...
\begin{enumerate}
\item A Fabry-Perot resonator consists of two mirrors separated by a
  distance $L$, each with an \emph{amplitude} reflectivity, $r$.

\item As we did for the single layer anti-reflection coating, we can
  write down the relationships between all the fields.

\item Assuming that the mirrors are moving very slowly, with respect
  to the round trip travel time for the light, we can find the
  quasi-static solution to the fields.

\item Take a few minutes to solve for the fields...
\end{enumerate}

As with the mass-spring and electrical (RLC) SHO, the optical
(laser-mirror) resonator also has a resonant frequency and a Q. Why?
\end{frame}


\begin{frame}
\frametitle{The Fabry-Perot Interferometer}

\centering
\includegraphics[width=0.99\textwidth]{../../Simulations/FabryPerot.pdf}

\end{frame}

\section{Summary}
\begin{frame}
\frametitle{Summary}
\begin{itemize}
  \item The interference of waves can be understood using superposition
  \item Multiple wave interference is a natural extension of this method
  \item Interferometers are precise measurement tools: can convert
    optical phase changes into measureable power.

  \item A diffraction grating gives a multi-slit diffraction pattern.
  \item For N slits, the angle between diffraction maxima is the
    wavelength over the slit spacing.
  \item By symmetry, the \emph{resolving} power (in radians) of a set
    of coherent receivers is set by the size of the receiver array.

\end{itemize}
\footnotetext[1]{\url{https://readingfeynman.org/2014/07/14/diffraction-gratings}}
\end{frame}

\end{document}

