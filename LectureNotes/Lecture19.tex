% Lecture for ph2a Caltech 2017: Vibrations and Waves
\documentclass[pdf, handout, xcolor={usenames,dvipsnames,svgnames,table}, hideothersubsections]{beamer}
\usepackage{beamerthemeshadow}
\mode<presentation>
  {
    \usefonttheme{structuresmallcapsserif}
    \usetheme{PaloAlto}
    \usecolortheme{whale}
    %\useinnertheme{circles}
    %\useoutertheme{tree}
  }

%\usepackage{svg}
\usepackage{xmpmulti}
\usepackage{mathtools}
%\usepackage{enumitem}
\usepackage{hyperref}
\hypersetup{
    pdffitwindow=true,     % window fit to page when opened
    colorlinks=true,       % false: boxed links; true: colored links
    linkcolor=Indigo,      % color of internal links
    citecolor=green,       % color of links to bibliography
    filecolor=magenta,     % color of file links
    urlcolor=blue,         % color of external links
    pdfstartview={Fit}
}

% Fonts/encoding
\renewcommand{\UrlFont}{\ttfamily\tiny}
\usepackage[utf8]{inputenc}
\usepackage[T1]{fontenc}
%\usepackage[sc,medium,raggedright]{titlesec}
\usepackage{newtxmath}
%\usepackage{libertine}
\usepackage[osf]{ebgaramond}

\graphicspath{{Figures/}}

\begin{document}
\title{Final Review}
\author{Caltech: ph2a}
\date{29 - Nov - 2017}


\frame{\titlepage}

\frame{\frametitle{Table of Contents}\tableofcontents}

\setbeamerfont{footnote}{size=\tiny}


\section{Driven Damped SHO}
   \subsection{Resonance}
%   \subsection{Complex Exponentials}
   \subsection{Frequency Response  (Bode) Plots}

\section{Fourier Analysis}
   \subsection{Harmonics}

\section{Coupled Oscillators}
   \subsection{Normal Modes}
\begin{frame}
\frametitle{Stabilizing Tall Buildings}

\centering
\includegraphics[width=0.99\textwidth]{Tallest_buildings_in_the_world.pdf}
\end{frame}

\begin{frame}
\frametitle{Stabilizing Tall Buildings}

\centering
\includegraphics[width=0.99\textwidth]{Tallest_buildings_in_the_world.pdf}
\end{frame}


\begin{frame}
\frametitle{Stabilizing Tall Buildings}
\begin{itemize}
\item Doha, Dubai; Buildings are tall, they shake too much
\item We put a Giant pendulum on top to damp it
\item What are the normal modes?
\item What mass and frequency and Q for the pendulum?
\end{itemize}
\end{frame}

\section{Waves}
   \subsection{Waves in different Media}
       \subsubsection{Strings, Sound / Gas, Water / Solids}
       \subsubsection{Electro-Magnetism}
\begin{frame}
\frametitle{Inertial Guidance}
In order to shoot a missile from Pyongyang to Los Angeles, the missile
should hit downtown within 1\,km. GPS will be jammed during wartime,
so one would need some kind of self-contained inertial guidance.\\
\includegraphics[width=0.5\textwidth]{NKorea2California.png} \\
Distance $\simeq$ 10000\,km. v = 6 km/s. $\theta = 1/10000$
\end{frame}
\begin{frame}
\frametitle{The Sagnac Interferometer}

\centering
\includegraphics[width=0.65\textwidth]{SagnacPhase.PNG}

\end{frame}
\subsection{Dispersion}
\subsection{Refraction}
\begin{frame}
\frametitle{Cold Lake: Sonic Lens}
\includegraphics[width=0.5\textwidth]{wasa.jpg} \\

Why is the lake cold? \\
Speed of sound as a function of temperature? \\
the speed gradient is like the index gradient in lenses.
\end{frame}
\subsection{Reflections}

\section{Interference}
   \subsection{Diffraction}

%\section{Summary}
\begin{frame}
\frametitle{Summary}
\begin{itemize}
  \item Study 1-2 examples of each Overview bullet
  \item Show all your work -> get partial credit!
  \item Draw diagrams

  \item Right clearly and legibly; making tiny equations that are
    unreadable is no help for your grade.


\end{itemize}
\end{frame}

\end{document}
