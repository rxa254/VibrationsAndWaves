% \documentclass[10pt]{article}
% \usepackage{amsmath}
% \usepackage{amssymb}
% \usepackage[margin=1in]{geometry}
% \setlength\parindent{0pt}
%
% \begin{document}
% \noindent

\section{Dispersion of Matter Waves}
An equation which is closely related to the wave equation is the Schrodinger equation:

$$i\frac{\partial\psi(x,t)}{\partial t}=-\frac{1}{2m}\frac{\partial^{2}\psi(x,t)}{\partial x^{2}},$$

where $m$ is a positive real number. Here $\psi(x,t)$ is a \it wavefunction \rm that describes the movement of a quantum mechanical matter wave with mass $m$. Much like the ordinary wave equation, the Schrodinger equation also admits traveling wave solutions of the form

$$\psi_{p}(x,t)=e^{ip\left(x-\frac{pt}{2m}\right)},$$

where $p\in\mathbb{R}$ is an arbitrary real number.

\begin{enumerate}[label=\alph*)]
\item (3 pts) What is the dispersion relation for the wave solutions of the Schrodinger equation? Give a brief sketch.


\item (2 pts) What is the phase velocity of these traveling waves?

\item (3 pts) Let us consider beats formed using a superposition of different wave solutions with similar frequencies. Let $p_{0}$ be a given real number. Consider a superposition of the form

$$\xi(x,t)=\psi_{p_{0}-\Delta p}(x,t)+\psi_{p_{0}}(x,t)+\psi_{p_{0}+\Delta p}(x,t),$$

where $\Delta p$ is some small number such that $\Delta p\ll p_{0}$. To linear order in $\Delta p$, show that we have

$$\xi(x,t)=\left(2\cos\left[\left(x-\frac{p_{0}}{m}t\right)\Delta p\right]+1\right)\psi_{p_{0}}(x,t).$$
\\

The previous result shows that, just like in the classical phenomenon of beats, a superposition of traveling matter waves with similar frequencies will give you an “envelope” function that modulates $\psi_{k_{0}}$. A reasonable definition of the group velocity in this case would be the velocity of the envelope function.


\item (2 pts) What is the group velocity of the superposition? Does this match up with what you expected from your knowledge of group velocities for classical waves?
\\

As an aside, the values of p actually correspond to the momentum of the quantum mechanical wave. With this interpretation, you may find your expression for the group velocity to be very familiar.

% \item (2 pts) Now consider these waves to be trapped in a box from $x=0$ to $x=L$. Inside the box, a right-moving wave will bounce off the wall to become a left-moving wave and vice versa. The waves inside the box must therefore be a general combination of a left-moving wave and a right-moving wave which will form a standing wave:
%
% $$f(x,t)=Ae^{ip\left(x-\frac{pt}{2m}\right)}+Be^{-ip\left(x+\frac{pt}{2m}\right)}.$$
%
% Since the particle cannot exit the box, quantum mechanics requires that the above superposition go to zero at the boundaries, i.e., $f(x=0,t)=0$ and $f(x=L,t)=0$. Up to an overall constant, find all the linear combinations which form valid quantum mechanical standing waves. What are the allowed values of $p$?
\end{enumerate}

 % Notice that the act of putting the waves into a box severely restricts the possible values of $p$ the standing wave is allowed to have. The previously unrestricted value of $p$ is now reduced to a discrete set of values. Thus we say that the act of restricting the wave \it quantizes \rm its momentum (and hence also energy). This is a very generic phenomenon in quantum mechanics.
%\end{document}
