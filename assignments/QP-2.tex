\documentclass[12pt]{article}
%\setlength{\oddsidemargin}{0in}
%\setlength{\topmargin}{0.0in}
%\setlength{\textwidth}{6.7in}
%\setlength{\textheight}{8.5in}
%
\usepackage{graphicx}

\usepackage{enumitem}

\usepackage{mathtools}
\usepackage[usenames,dvipsnames]{xcolor}

\usepackage{geometry}
 \geometry{
 letterpaper,
 textwidth=6.5in
 }
\usepackage[utf8]{inputenc}
\usepackage{libertine}
\usepackage{libertinust1math}
%\usepackage[libertine,cmintegrals,cmbraces,vvarbb]{newtxmath}
%\usepackage{newtxmath}
%\usepackage[osf]{ebgaramond}
\usepackage[T1]{fontenc}
%\usepackage{palatino}
\usepackage{microtype}

\usepackage{fancyhdr}

\pagestyle{fancy}
\fancyhf{}
\fancyhead[RE,LO]{\textcolor{BlueViolet}{Vibrations and Waves}}
\fancyhead[LE,RO]{ph2a: 2017}
\fancyfoot[RE,LO]{\textcolor{Orange}{Caltech}}
\fancyfoot[LE,RO]{\textcolor{Orange}{Physics, Math, and Astronomy}}

%\input mydefs.tex
\def\vev#1{\left\langle #1\right\rangle}
\def\hb{\hfill\break}
%
\begin{document}
%
\begin{centering}
\LARGE{QP 2: Triangle Wave Pulse}
\end{centering}
\bigskip
\bigskip

A string with length $L$, tension $T$, and mass per unit length $\mu$,
is fixed at both ends.

\begin{enumerate}[label=(\alph*)]
\item (1 point) What is the angular frequency, $\omega_1$,
of the fundamental mode ($n = 1$) ?

\item (1 point) Write down the most general series equation for the time, $t$, and space, $x$, dependence of the displacement, $y(x,t)$, of the string.

\item (2 points) The string is released at $t = 0$ with the following displacement:
\[
y(x , t = 0) = \left.
\begin{cases}
\frac{4 x h}{L}, & 0 < x < L/4 \\
2 h - \frac{4 x h}{L}, & L/4 < x < L/2 \\
0, & L/2 < x < L
\end{cases}
\right\}
\]
Write down an expression for the first \textbf{four} coefficients for $A_n$ ($n = 1,2,3,4$) in the Fourier series expansion for $y(x,t)$. Assume the string is released from rest.

\item (2 points) Are any of these first four coefficients zero? If so, describe why. (Hint: think about the symmetry of the eigenmodes with respect to the initial waveform)

\item (2 points) Sketch (by hand) or plot (using software) the approximation for $y(x, t=0)$ obtained by using only the first four terms in the series expansion.

\item (2 points) Sketch or plot the approximation for $y(x,t)$,
at $t = \pi/\omega_1$, obtained by using only the first four terms in the series expansion.

\end{enumerate}
\bigskip
{\color{Sepia} \hrule}



%\clearpage



\end{document}
