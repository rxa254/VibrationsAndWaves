\documentclass[12pt]{article}
%\setlength{\oddsidemargin}{0in}
%\setlength{\topmargin}{0.0in}
%\setlength{\textwidth}{6.7in}
%\setlength{\textheight}{8.5in}
%
\usepackage{graphicx}
\usepackage{enumitem}
\usepackage{mathtools}
\usepackage[usenames,dvipsnames]{xcolor}

\usepackage{geometry}
 \geometry{
 letterpaper,
 textwidth=6.5in
 }
\usepackage[utf8]{inputenc}
%\usepackage{libertine}
%\usepackage{libertinust1math}
\usepackage[libertine,cmintegrals,cmbraces,vvarbb]{newtxmath}
%\usepackage{newtxmath}
%\usepackage[osf]{ebgaramond}
\usepackage[T1]{fontenc}
%\usepackage{palatino}
\usepackage{microtype}

\usepackage{fancyhdr}

\pagestyle{fancy}
\fancyhf{}
\fancyhead[RE,LO]{\textcolor{BlueViolet}{Vibrations and Waves}}
\fancyhead[LE,RO]{ph2a: 2017}
\fancyfoot[RE,LO]{\textcolor{Orange}{Caltech}}
\fancyfoot[LE,RO]{\textcolor{Orange}{Physics, Math, and Astronomy}}

%\input mydefs.tex
\def\vev#1{\left\langle #1\right\rangle}
\def\hb{\hfill\break}
%
\begin{document}
%
\begin{centering}
\LARGE{QP 3: The Energy in the Waves}
\end{centering}
\bigskip
\bigskip

We want to show that the total energy of a string is the sum of the energies of each normal mode. The string has a length $L$ and is fixed at both ends.

\begin{enumerate}[label=(\alph*)]
\item (2 pts)
Consider a single normal mode with $\lambda_n = 2 L/n$. Take a time $t$ when the string is unstretched and all the energy is in the kinetic energy. Assuming that the mode has a maximum amplitude at $t = 0$, at what time is all the energy in the kinetic energy? Write down an expression for the total kinetic energy and evaluate it at this time.

\item (2 pts)
Assuming all of the initial phases $\phi_n = 0$, write down the general expression for an arbitrary disturbance on the string (including all possible modes).

\item (2 pts)
Write down an expression for the total kinetic energy for this disturbance.

\item (2 pts)
Noting that the integral will be over common terms and cross terms, use Fourier analysis (i.e. orthogonality of the eigenfunctions) to show that the result will be a sum of the individual kinetic energies for each mode.

\end{enumerate}
\bigskip
{\color{Sepia} \hrule}



%\clearpage



\end{document}
