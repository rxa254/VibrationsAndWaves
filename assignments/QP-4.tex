\documentclass[12pt]{article}
%\setlength{\oddsidemargin}{0in}
%\setlength{\topmargin}{0.0in}
%\setlength{\textwidth}{6.7in}
%\setlength{\textheight}{8.5in}
%
\usepackage{graphicx}
\usepackage{enumitem}
\usepackage{mathtools}
\usepackage[usenames,dvipsnames]{xcolor}

\usepackage{geometry}
 \geometry{
 letterpaper,
 textwidth=6.5in
 }
\usepackage[utf8]{inputenc}
%\usepackage{libertine}
%\usepackage{libertinust1math}
\usepackage[libertine,cmintegrals,cmbraces,vvarbb]{newtxmath}
%\usepackage{newtxmath}
%\usepackage[osf]{ebgaramond}
\usepackage[T1]{fontenc}
%\usepackage{palatino}
\usepackage{microtype}

\usepackage{fancyhdr}

\pagestyle{fancy}
\fancyhf{}
\fancyhead[RE,LO]{\textcolor{BlueViolet}{Vibrations and Waves}}
\fancyhead[LE,RO]{ph2a: 2017}
\fancyfoot[RE,LO]{\textcolor{Orange}{Caltech}}
\fancyfoot[LE,RO]{\textcolor{Orange}{Physics, Math, and Astronomy}}

%\input mydefs.tex
\def\vev#1{\left\langle #1\right\rangle}
\def\hb{\hfill\break}
%
\begin{document}
%
\begin{centering}
\LARGE{QP 4: Interstellar Dispersion}
\end{centering}
\bigskip
\bigskip

Several Nobel prizes in physics have been given out for discoveries made with pulsars. A neutron star is an extremely dense remnant of a star, after that star has burned up all the nuclear fuel that makes it shine. It has a mass comparable to the mass of the Sun, but occupies a space about the size of Pasadena.
A pulsar is a rapidly spinning neutron star; they may spin as fast as 1000 times per second.
Pulsars emit extremely short bursts of radiation at regular intervals, every
time they spin around. We see these as periodic radio signals with a period
equal to the spin period of the pulsar.
Consider now a single burst of radio emission from a pulsar at a distance
D$ = 3 \times 10^{20}$\,m away from us. The radio signal from the pulsar has to
propagate through interstellar space before reaching us. That interstellar
space is filled with a plasma; a gas of ionized atoms with an electron density
$n_e = 2 \times 10^{-8}~\mathrm{m}^{-3}$.
As you know, electromagnetic waves propagate through a vacuum at the
speed of light, but if they propagate through a medium, they may travel a
bit slower. Electromagnetic waves in a plasma have a dispersion relation,
\begin{equation}
\omega^2 = \omega_p^2 + k^2 c^2
\end{equation}
connecting the (angular) frequency, $\omega$, of the wave and its wavenumber, $k = 2 \pi / \lambda$, where $\lambda$ is the wavelength. Here, $\omega_p$ is the (angular) "plasma frequency" given by:
\begin{equation}
\omega_p^2 = \frac{n_e e^2}{m_e \epsilon_0}
\end{equation}
where $e = 1.6 \times 10^{-19} C$ is the electron charge, $m_e = 9.11 \times 10^{-31}$~kg is the electron mass, and $\epsilon_0 = 8.85 \times 10^{-12}$~F/m is the permittivity of the vacuum.
\begin{enumerate}[label=(\alph*)]
\item (2 pts)
Evaluate the plasma frequency, $\omega_p$, for interstellar space.

\item (2 pts)
Write down an expression for the phase velocity, $v_{ph}$, in terms of the
wavenumber, $k$.

\item (2 pts)
Write down an expression for the group velocity, $v_{g}$, in terms of the
wavenumber, $k$.

\item (2 pts)
Suppose the pulsar emits simultaneous pulses centered at two different
(angular) frequencies: $\omega_1 = 1$\,GHz and $\omega_2 = 2$\,GHz.
These will now propagate through interstellar space at slightly
different velocities. Calculate the difference in the arrival times of these
two pulses.

\end{enumerate}
\bigskip
{\color{Sepia} \hrule}



%\clearpage



\end{document}
