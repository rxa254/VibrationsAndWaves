\documentclass[12pt]{article}
%\setlength{\oddsidemargin}{0in}
%\setlength{\topmargin}{0.0in}
%\setlength{\textwidth}{6.7in}
%\setlength{\textheight}{8.5in}
%
\usepackage{graphicx}
\usepackage{enumitem}
\usepackage{mathtools}
\usepackage[usenames,dvipsnames]{xcolor}

\usepackage{geometry}
 \geometry{
 letterpaper,
 textwidth=6.5in
 }
\usepackage[utf8]{inputenc}
%\usepackage{libertine}
%\usepackage{libertinust1math}
\usepackage[libertine,cmintegrals,cmbraces,vvarbb]{newtxmath}
%\usepackage{newtxmath}
%\usepackage[osf]{ebgaramond}
\usepackage[T1]{fontenc}
%\usepackage{palatino}
\usepackage{microtype}

\usepackage{fancyhdr}
\usepackage[colorlinks=true]{hyperref}

\pagestyle{fancy}
\fancyhf{}
\fancyhead[RE,LO]{\textcolor{BlueViolet}{Vibrations and Waves}}
\fancyhead[LE,RO]{ph2a: 2017}
\fancyfoot[RE,LO]{\textcolor{Orange}{Caltech}}
\fancyfoot[LE,RO]{\textcolor{Orange}{Physics, Math, and Astronomy}}

%\input mydefs.tex
\def\vev#1{\left\langle #1\right\rangle}
\def\hb{\hfill\break}
%
\begin{document}
%
\begin{centering}
\LARGE{QP 6: Sound Waves in Water}
\end{centering}
\bigskip
\bigskip

A longitudinal sound wave propagates in air with velocity
c$_{\rm air} = 340$\,m/s, and in water with $c_{\rm water} = 1480$\,m/s.

\begin{enumerate}[label=(\alph*)]
\item (2 pts)
Suppose that sound approaches the surface of a pool of water at an angle $\theta_1$ with respect to
the normal. Some sound will be transmitted into the pool. At what angle $\theta_2$ (again with respect to
the normal) does the transmitted wave travel in the water?

\item (2 pts)
Could total internal reflection happen when we whistle from the shore of a lake towards the
surface? Could it happen when we whistle from underwater facing the surface? At what angles?

\item (1 point)
Is there a specific case when the reflected sound wave has a polarization parallel to the
interface? Explain.

\item (2 pts)
Consider a sound wave normally incident to the surface from water to air. What are the
reflection and transmission coefficients for this wave? What continuity equations did you use to derive
these formulas?

\item (2 pts)
In a different setup, you're in a boat on the surface of the lake and whistle towards a diving
friend whose head is underwater. What frequency will she hear if she's stationary? What frequency
will she hear if she swims towards the surface (i.e. towards you) at velocity $v = 1.5$\,m/s?
And what
frequency when she's stationary but the water is moving steadily towards you (and away from her)
with a velocity of $u = 0.5$\,m/s? Your voice has a frequency of $f = 200$\,Hz.

\end{enumerate}
\bigskip
{\color{Sepia} \hrule}







\end{document}
