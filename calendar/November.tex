%%%%%%%%%%%%%%%%%%%%%%%%%%%%%%%%%%%%%%%%%
% Monthly Calendar
% LaTeX Template
%
% This template has been downloaded from:
% http://www.latextemplates.com
%
% Original calendar style author:
% Evan Sultanik (http://www.sultanik.com/LaTeX_calendar_style)
%
% Important note:
% This template requires the calendar.sty file to be in the same directory as the
% .tex file. The calendar.sty file provides the necessary structure to create the
% calendar.
%
%%%%%%%%%%%%%%%%%%%%%%%%%%%%%%%%%%%%%%%%%

%----------------------------------------------------------------------------------------
%	PACKAGES AND OTHER DOCUMENT CONFIGURATIONS
%----------------------------------------------------------------------------------------

\documentclass[landscape,a4paper]{article}

\usepackage{calendar} % Use the calendar.sty style

\usepackage[landscape,margin=0.5in]{geometry}
% Fonts/encoding
\usepackage[utf8]{inputenc}
\usepackage[T1]{fontenc}
\usepackage[sc,medium,raggedright]{titlesec}
\usepackage{newtxmath}
%\usepackage[osf]{ebgaramond}
\usepackage{librebaskerville}

% Hyperref wants to be loaded last of all...
\usepackage{hyperref}


\begin{document}

\pagestyle{empty} % Removes the page number from the bottom of the page

\noindent

% Calendar starting day, 
% default of 1 means Sunday, 2 for Monday, etc
\StartingDayNumber=1 

%----------------------------------------------------------------------------------------
%	MONTH AND YEAR SECTION
%----------------------------------------------------------------------------------------

\begin{center}
\textsc{\LARGE November - December}\\ % Month
\textsc{\large Caltech \href{https://piazza.com/caltech/fall2017/ph2a/home}{ph2a} 2017} % Year
\end{center}

%----------------------------------------------------------------------------------------

\begin{calendar}{\hsize}
 
%----------------------------------------------------------------------------------------
%	BLANK DAYS BEFORE THE BEGINNING OF THE CALENDAR
%----------------------------------------------------------------------------------------

% This part is very finicky. It defines the number of blank days at the beginning of the calendar before the first of the month starts. If you need this to be more than 4 (i.e. the first starts on a Friday or Saturday in a 31 day month), then you have two options: 
% 1) You can uncomment another one or two \BlankDay's below which will make a new week (6 total) which makes the calendar too big for one page, remedy this by decreasing the size of each day by replacing 2.5cm below with a smaller number. 
% 2) Make the spill-over days start at the top left of the calendar (i.e. the calendar starts with 31 then a few days blank then 1, 2, 3, etc). The second option can be configured by uncommenting the below:

%\setcounter{calendardate}{31} % Begin the count with 31 so the top left day is 31; this can be changed to 29 or 30 as required
%\day{}{\vspace{2.5cm}} % 31 - add another line identical to this if starting at 30 or earlier

% You will need to comment out the 31 in the NUMBERED DAYS AND CALENDAR CONTENT section below for this as well as commenting out one of the \BlankDay's below. Play around with it and you will get it.

%\BlankDay
%\BlankDay
%\BlankDay
%\BlankDay
%\BlankDay
%\BlankDay

%----------------------------------------------------------------------------------------
%	NUMBERED DAYS AND CALENDAR CONTENT
%----------------------------------------------------------------------------------------

% These are the numbered days in the template - if there are less than 31 days simply comment out the bottom lines.

% \vspace{2.5cm} is only there to provide an even look to the calendar where each day is 2.5cm tall, it can be changed or removed to automatically adjust to the day in the week with the most content

\setcounter{calendardate}{29} % Start the date counter at 1

\day{}{} % 29
\day{Reading:}{French: \\
               Ch. 6: 181-189 \\
               Ch. 7: 216-246} % 30
\day{Lecture 11:}{3D Waves and Music} % 31
\setcounter{calendardate}{1}
\day{Homework \#5 Due:}{Ch. 3: 11 \\
                      Ch. 6: 7, 8 \\
                      Ch. 7: 6 \\
                      QP--3} % 1
\day{Lecture 12:}{Dispersion} % 2
\day{Handout:}{Quiz \#3} % 3
\day{}{\vspace{2.5cm}} % 4

% DONT ASSIGN 8-6 !!

\day{}{\vspace{2.5cm}} % 5
\day{Reading:}{French: \\
               Ch. 8: 253-280} % 6
\day{Lecture 13:}{Reflection \& Transmission} % 7
\day{Homework \#6 Due}{Ch. 6: 13 \\
                       Ch. 7: 18, 20, 24 \\
                       QP--4} % 8
\day{Lecture 14:}{Polarization, Refraction, Doppler} % 9
\day{}{\textbf{\textcolor{Blue}{Quiz \#3 due}}} % 10
\day{}{\vspace{2.5cm}} % 11


\day{}{\vspace{2.5cm}} % 12
\day{Reading:}{Lecture Notes} % 13
\day{Lecture 15:}{Optics} % 14
\day{Homework \#7 Due}{Ch. 7: 25 \\
                       Ch. 8: 1, 5 \\
                       QP--5 }% 15
\day{Lecture 16:}{Applications in Telecom} % 16
\day{Handout:}{Quiz \#4} % 3
\day{}{\vspace{2.5cm}} % 18


\day{}{\vspace{2.5cm}} % 19
\day{Reading:}{French: \\
               Ch. 8: 280\,--\,284} % 20
\day{Lecture 17:}{Interference} % 21
\day{Homework \#8 Due}{Ch. 8: 8, 9, 12 \\
                       QP--6} % 22
\day{Thanksgiving}{Recess} % 23
\day{Thanksgiving}{Recess} % 24
\day{}{\vspace{2.5cm}} % 25


\day{}{\vspace{2.5cm}} % 26
\day{Reading:}{French: \\
               Ch. 8: 284\,--\,298} % 27
\day{Lecture 18:}{Diffraction \\
  \textbf{\textcolor{Blue}{Quiz \#4 due}}} % 28
\day{Homework \#9 Due}{Ch. 8: 14, 15 \\
                       QP--7, QP--8} % 29
\day{Lecture 19:}{Review} % 30
\setcounter{calendardate}{1}
\day{Handout :}{Final Exam \\
  Due Dec. 9} % 1
\day{}{\vspace{2.5cm}} % 2




\finishCalendar
\end{calendar}
\end{document}
